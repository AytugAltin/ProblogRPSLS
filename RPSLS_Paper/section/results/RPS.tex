\subsection{rock paper scissors (RPS)}
The first model is a simple rock paper scissors solver that given two images decides the winner. The possible options are tie, player 1 wins, player 2 wins. Example images, source \cite{RPSLS-database}, can be found in figure \ref{fig:rps_input}. There are 1440 training examples and 360 examples are used to calculate the model's accuracy. 

\begin{figure}[htp]
    \centering
    \includegraphics[width=.3\textwidth]{figures/input/paper.jpg}\hfill
    \includegraphics[width=.3\textwidth]{figures/input/rock.jpg}\hfill
    \includegraphics[width=.3\textwidth]{figures/input/scissors.jpg}
    \caption{Example of input pictures with hand gestures: paper, rock and scissors\cite{RPSLS-database}.} %%TODO import ref of pictures source
    \label{fig:rps_input}
\end{figure}

\paragraph{Implementation details:} For the DeepProbLog implementation, a similar approach as \cite{DBLP} has been taken:
\begin{itemize}
    \item Cross-entropy loss between predicted and desired outcomes
    \item The network architecture: 2 convolutional layers with kernel size 5, and respectively 6 and 16 filters, both followed with a maxpool-layer of size 2 and stride 2 which are also both followed by the activation layer ReLU. These are followed by 3 linear layers 120, 84 and 3, The first 2 layers are followed by the activation layer ReLU and the last by a softmax layer. \item The learning rate has been set to 0.0001. 
    \item Adam \cite{kingma2014adam} optimization for the neural networks, SGD for the logic parameters is used.
  \end{itemize}
  On listing \ref{lst:rps-logic} the logical model can be found for the DeepProbLog model.

  \begin{lstlisting}[label={lst:rps-logic},language=Prolog,frame=single,caption={Rock paper scissors DeepProbLog model},captionpos=b]
    nn(rps_net,[X],Y,[paper,scissors,rock]) :: sign(X,Y).

    rps(X,Y,0) :- sign(X,Z), sign(Y,Z).

    rps(X,Y,1) :- sign(X,paper), sign(Y,rock).
    rps(X,Y,2) :- sign(X,paper), sign(Y,scissors).
    rps(X,Y,2) :- sign(X,rock), sign(Y,paper).
    rps(X,Y,1) :- sign(X,rock), sign(Y,scissors).
    rps(X,Y,1) :- sign(X,scissors), sign(Y,paper).
    rps(X,Y,2) :- sign(X,scissors), sign(Y,rock).
    \end{lstlisting}

    The implementation and architecture for the CNN that we use to compare the DeepProbLog network are similar to the DeepProbLog model. However, the outcomes of the last layer do represent the winner (3 options) instead of the hand gesture (also 3 options).


\subsubsection{Results}
The loss and accuracy over the iterations of both models are plotted in figure \ref{fig:rps_output}. We can clearly see that the DeepProbLog has an advantage over the CNN. The CNN reaches 100\% accuracy over 1350 iterations in 24.2 seconds while the DeepProbLog example reaches it in 350 iterations in 18.7 seconds. 


\begin{figure}[h]
    \centering
    \begin{subfigure}[b]{0.49\textwidth}
        \begin{tikzpicture}
            \begin{axis}[xlabel=Iterations,ylabel=Loss]
                \addplot[thin,red] table [x=i, y=loss, col sep=comma] {results/RPS/RPS_BaseLine_loss.log};
                \addplot[thin,blue]  table [x=i, y=loss, col sep=comma] {results/RPS/RPS_Problog_loss.log};
                
            \end{axis}
        \end{tikzpicture}
        \caption{Loss of both networks over number of iterations}
    \end{subfigure}
    \hfill
    \begin{subfigure}[b]{0.49\textwidth}
        \begin{tikzpicture}
            \begin{axis}[xlabel=Iterations,ylabel=Accuracy]
                \addplot[thin,red] table [x=i, y=Accuracy, col sep=comma] {results/RPS/RPS_BaseLine_accuracy.log};
                \addplot[thin,blue]  table [x=i, y=Accuracy, col sep=comma] {results/RPS/RPS_Problog_accuracy.log};
                
            \end{axis}
        \end{tikzpicture}
        \caption{Accuracy of both networks over number of iterations}
    \end{subfigure}
    \caption{Performance (loss and accuracy) of both networks over number of iterations: blue for DeepProbLog, red for CNN}
    \label{fig:rps_output}
\end{figure}  
