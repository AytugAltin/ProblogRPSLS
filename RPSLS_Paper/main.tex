\documentclass[10pt,letterpaper]{article}
\usepackage{cite}
%% Welcome to Overleaf!
%% If this is your first time using LateX, it might be worth going through this brief presentation:
%% https://www.overleaf.com/latex/learn/free-online-introduction-to-latex-part-1

%% Researchers have been using LateX for decades to typeset their papers, producing beautiful, crisp documents in the process. By learning LateX, you are effectively following in their footsteps, and learning a highly valuable skill!

%% The \usepackage commands below can be thought of as analogous to importing libraries into Python, for instance. We've pre-formatted this for you, so you can skip right ahead to the title below.

%% Language and font encodings
\usepackage[spanish,english]{babel}
\usepackage[utf8x]{inputenc}
\usepackage[T1]{fontenc}

%% Sets page size and margins
\usepackage[a4paper,top=3cm,bottom=2cm,left=3cm,right=3cm,marginparwidth=1.75cm]{geometry}

%% Useful packages
\usepackage{amsmath}
\usepackage{graphicx}
\usepackage[colorinlistoftodos]{todonotes}
\usepackage[colorlinks=true, allcolors=blue]{hyperref}
\usepackage{float}
\usepackage{tikz}
\usepackage{pgfplots}
\usepackage{subcaption}
\usepackage{listings}
\usepackage{multirow}
\pgfplotsset{compat=newest}
\usepgfplotslibrary{units}


%% Title
\title{
		%\vspace{-1in} 	
		\usefont{OT1}{bch}{b}{n}
		\normalfont \normalsize \textsc{Ku leuven} \\ [10pt]
		\huge DeepProbLog Rock Paper Scissors\\
}

\usepackage{authblk}
\author[0]{Aytug Altin \\
		Webber Academy \\
        \textit{*Both authors contributed equally to the writing and research in this study. Their names are listed in alphabetical order.}}

\begin{document}
\maketitle

\selectlanguage{english}
\begin{abstract}
	abtests abstract abstract abstract abstract abstract abstract abstract abstract abstract abstract abstract abstract abstract abstract abstract abstract abstract abstract abstract abstract abstract abstract abstract abstract abstract abstract abstract abstract abstract abstract abstract abstract abstract abstract abstract abstract abstract abstract abstract abstract abstract abstract abstract abstract
\end{abstract}
{\textbf{Keywords}
Keyword-Keyword; Keyword; Keyword Keyword; Keyword-Keyword Keyword}


\section{Introduction}
DeepProbLog \cite{DBLP} can be used to incorporate logical predicates into neural network model. Here, we try to compare the a DeepProbLog model with a normal CNN by training it on the simple game rock paper scissors. Testing if the DeepProbLog model has an advantage over the normal CNN is the goal of this experiment. By increasing the complexity of the game, which is done by adding 2 possible gestures lizard and spock, the effect on the performance of both models can be compared. We want to see what hapenes when the complexity increases, how both models handle it and if the DeepProbLog model is more scalable. We believe that by incorporating the logical predicate we give the model additional information of rules which will result in a faster learning process compared to a CNN that does not have any understanding of the rules. The results are interesting to see how big of a difference these logical models have over the default CNN, it can become a new era of research and a new technique to optimize existing models.


\section{Experimental Evaluation}
In this section the models are introduced and their results are being evaluated. Images of 50 by 50 pixels are used as input. All runs are executed on CPU. The train ratio is 0.8 and the validation set that is used to compute the accuracy is 0.2


\subsection{rock paper scissors (RPS)}
The first model is a simple rock paper scissors solver that given two images decides the winner. The possible options are: tie, player 1 wins, player 2 wins. Example images, source \cite{RPSLS-database}, can be found on figure \ref{fig:rps_input}. There are 1440 training examples and 360 examples are used to calculate the model's accuracy. 

\begin{figure}[htp]
    \centering
    \includegraphics[width=.3\textwidth]{figures/input/paper.jpg}\hfill
    \includegraphics[width=.3\textwidth]{figures/input/rock.jpg}\hfill
    \includegraphics[width=.3\textwidth]{figures/input/scissors.jpg}
    \caption{Example of input pictures with hand gestures: paper, rock and scissors\cite{RPSLS-database}.} %%TODO import ref of pictures source
    \label{fig:rps_input}
\end{figure}

\paragraph{Implementation details:} For the DeepProbLog implementation, a similar approach as \cite{DBLP} has been taken:
\begin{itemize}
    \item Cross-entropy loss between predicted and desired outcomes
    \item The network architecture: 2 convolutional layers with kernel size 5, and respectively 6 and 16 filters, both followed with a maxpool-layer of size 2 and stride 2 which are also both followed by the activation layer ReLU. These are followed by 3 linear layers 120, 84 and 3, The first 2 layers are followed by the activation layer ReLU and the last by a softmax layer. \item The learning rate has been set to 0.0001. 
    \item Adam \cite{kingma2014adam} optimization for the neural networks, and SGD for the logic parameters is used.
  \end{itemize}
  On listing \ref{lst:rps-logic} the logical model can be found for the DeepProbLog model.

  \begin{lstlisting}[label={lst:rps-logic},language=Prolog,frame=single,caption={Rock paper scissors DeepProbLog model},captionpos=b]
    nn(rps_net,[X],Y,[paper,scissors,rock]) :: sign(X,Y).

    rps(X,Y,0) :- sign(X,Z), sign(Y,Z).

    rps(X,Y,1) :- sign(X,paper), sign(Y,rock).
    rps(X,Y,2) :- sign(X,paper), sign(Y,scissors).
    rps(X,Y,2) :- sign(X,rock), sign(Y,paper).
    rps(X,Y,1) :- sign(X,rock), sign(Y,scissors).
    rps(X,Y,1) :- sign(X,scissors), sign(Y,paper).
    rps(X,Y,2) :- sign(X,scissors), sign(Y,rock).
    \end{lstlisting}

    The implementation and architecture for the CNN that we use to compare the DeepProbLog network is similar to the DeepProbLog model. However, outcomes of the last layer does represent the winner (3 options) instead of the hand gesture (also 3 options).


\subsubsection{Results}
The loss and accuracy over the iterations of both models are plotted on figure \ref{fig:rps_output}. We can clearly see that the DeepProbLog has an advantage over the CNN. The CNN reaches 100\% accuracy over 1350 iterations in 24.2 seconds while the DeepProbLog example reaches it in 350 iterations in 18.7 seconds. 


\begin{figure}[h]
    \centering
    \begin{subfigure}[b]{0.49\textwidth}
        \begin{tikzpicture}
            \begin{axis}[xlabel=Iterations,ylabel=Loss]
                \addplot[thin,red] table [x=i, y=loss, col sep=comma] {results/RPS/RPS_BaseLine_loss.log};
                \addplot[thin,blue]  table [x=i, y=loss, col sep=comma] {results/RPS/RPS_Problog_loss.log};
                
            \end{axis}
        \end{tikzpicture}
        \caption{Loss of the both networks over number of iterations}
    \end{subfigure}
    \hfill
    \begin{subfigure}[b]{0.49\textwidth}
        \begin{tikzpicture}
            \begin{axis}[xlabel=Iterations,ylabel=Accuracy]
                \addplot[thin,red] table [x=i, y=Accuracy, col sep=comma] {results/RPS/RPS_BaseLine_accuracy.log};
                \addplot[thin,blue]  table [x=i, y=Accuracy, col sep=comma] {results/RPS/RPS_Problog_accuracy.log};
                
            \end{axis}
        \end{tikzpicture}
        \caption{Accuracy of the both networks over number of iterations}
    \end{subfigure}
    \caption{Performance (loss and accuracy) of the both networks over number of iterations: blue for DeepProbLog, red for CNN}
    \label{fig:rps_output}
\end{figure}  


\subsection{rock paper scissors lizard spock(RPS-LS)}

This model is a bit more advanced. The possible options are still the same, however, there are 2 more possible inputs: spock and lizard, these are visible figure \ref{fig:rpsls_input}. 

\begin{figure}[htp]
    \centering
    \includegraphics[width=.3\textwidth]{figures/input/lizard.jpg}
    \includegraphics[width=.3\textwidth]{figures/input/spock.jpg}
    \caption{Example of input pictures with hand gestures: lizard and spock\cite{RPSLS-database}.} %%TODO import ref of pictures source
    \label{fig:rpsls_input}
\end{figure}

On listing \ref{lst:rpsls-logic} the model can be found that is used, it is similar to the RPS model.

\begin{lstlisting}[label={lst:rpsls-logic},language=Prolog,frame=single,caption={Rock paper scissors lizard spock DeepProbLog model},captionpos=b]
nn(rpsls_net,[X],Y,[paper,scissors,rock,lizard,spock]) :: sign(X,Y).

rpsls(X,Y,0) :- sign(X,Z), sign(Y,Z).

rpsls(X,Y,1) :- sign(X,paper), sign(Y,rock).
rpsls(X,Y,2) :- sign(X,paper), sign(Y,scissors).
rpsls(X,Y,2) :- sign(X,paper), sign(Y,lizard).
rpsls(X,Y,1) :- sign(X,paper), sign(Y,spock).

rpsls(X,Y,1) :- sign(X,scissors), sign(Y,paper).
rpsls(X,Y,2) :- sign(X,scissors), sign(Y,rock).
rpsls(X,Y,1) :- sign(X,scissors), sign(Y,lizard).
rpsls(X,Y,2) :- sign(X,scissors), sign(Y,spock).

rpsls(X,Y,1) :- sign(X,rock), sign(Y,scissors).
rpsls(X,Y,2) :- sign(X,rock), sign(Y,paper).
rpsls(X,Y,1) :- sign(X,rock), sign(Y,lizard).
rpsls(X,Y,2) :- sign(X,rock), sign(Y,spock).

rpsls(X,Y,2) :- sign(X,lizard), sign(Y,scissors).
rpsls(X,Y,1) :- sign(X,lizard), sign(Y,paper).
rpsls(X,Y,2) :- sign(X,lizard), sign(Y,rock).
rpsls(X,Y,1) :- sign(X,lizard), sign(Y,spock).

rpsls(X,Y,1) :- sign(X,spock), sign(Y,scissors).
rpsls(X,Y,2) :- sign(X,spock), sign(Y,paper).
rpsls(X,Y,1) :- sign(X,spock), sign(Y,rock).
rpsls(X,Y,2) :- sign(X,spock), sign(Y,lizard).
    \end{lstlisting}


\paragraph{Implementation details:} For the DeepProbLog implementation, a similar approach has been taken as the RPS model. The model now has 5 outputs instead of 3 because of the addition of 2 new hand gestures: lizard and spock. The first baseline CNN does still has the same number of outputs: tie, player1 wins and player2 wins. 

\subsubsection{Results}
The loss and accuracy over the iterations of both models are plotted on figure \ref{fig:rpsls_output_loss} and figure \ref{fig:rpsls_output_acc} respectively. Notice that the scales are different from those we have seen on figre \ref{fig:rps_output}. We can clearly see by keeping the same models and increasing the complexity, the CNN model struggles more. It takes the CNN model 38000 iterations to reach an accuracy of 100\% while the DeepProbLog model takes only 3700 iterations to reach the same level of accuracy. The CNN takes up to 661 seconds until it reaches this accuracy. After excessive tuning and trying different hyperparameters for this CNN model \footnote{These tuning methods include altering the learning rate, gradually increasing or decreasing the learning rate \cite{lr-decay}, gradually increasing the batch-size \cite{increase-batch}, choosing different optimizer, L2 regularisation, altering the CNN structure (adding layers and/or nodes)}, the model was able to achieve a 100\% accuracy a little faster.  Doubling the learning rate from $0.0001$ to $0.0002$, the new CNN model was able to achieve in 23000 iterations and in 399.2 seconds. 


\begin{figure}
    \begin{tikzpicture}[yscale=0.9,xscale=0.9]
        \begin{axis}[xlabel=Iterations,ylabel=Loss,xscale=2]
            \addplot[thin,red] table [x=i, y=loss, col sep=comma] {results/RPSLS/RPSLS_BaseLine_loss.log};
            \addplot[thin,blue]  table [x=i, y=loss, col sep=comma] {results/RPSLS/RPSLS_Problog_loss.log};
            \addplot[ultra thin,green]  table [x=i, y=loss, col sep=comma] {results/RPSLS/RPSLS_BaseLine_losslr0.0002lrm1.log};
        \end{axis}
    \end{tikzpicture}
    \caption{Loss of RPS-LS networks over number of iterations}
    \label{fig:rpsls_output_loss}
\end{figure}
\begin{figure}
    \begin{tikzpicture}[yscale=0.9,xscale=0.9]
        \begin{axis}[xlabel=Iterations,ylabel=Accuracy,xscale=2]
            \addplot[thin,red] table [x=i, y=Accuracy, col sep=comma] {results/RPSLS/RPSLS_BaseLine_accuracy.log};
            \addplot[thin,blue]  table [x=i, y=Accuracy, col sep=comma] {results/RPSLS/RPSLS_Problog_accuracy.log};
            \addplot[ultra thin,green]  table [x=i, y=Accuracy, col sep=comma] {results/RPSLS/RPSLS_BaseLine_accuracylr0.0002lrm1.log};
        \end{axis}
    \end{tikzpicture}
    \caption{Accuracy of RPS-LS networks over number of iterations}
    \label{fig:rpsls_output_acc}
\end{figure}








\subsection{Final results}


\begin{table}[!h]
    \centering
    \begin{tabular}{|c|c|c|c|}
        \hline
        Model & iterations until 100\% accuracy  & time ( until 100\% accuracy    \\
        \hline\hline
        RPS CNN & 1350 & 24.2 s \\
        \hline
        \textbf{RPS DeepProbLog} & \textbf{350} & \textbf{18.7 s} \\
        \hline
        \hline
        RPS-LS CNN & 38000 & 661.2 s\\
        \hline
        RPS-LS CNN (tuned lr=0.0002)& 23400 & 399.2 s\\
        \hline
        \textbf{RPS-LS DeepProbLog} & \textbf{3700} & \textbf{246.2 s}\\
        \hline
    \end{tabular}
    \caption{Results of the different models}
    \label{t1}
    \end{table}




\section*{Conclusions}
concluded


\section*{Acknowledgements}
We would like to extend our sincerest gratitude and appreciation to everyone who helped make this project a possibility.Acknowledgements Acknowledgements Acknowledgements Acknowledgements Acknowledgements Acknowledgements Acknowledgements Acknowledgements

\bibliographystyle{plain}
\bibliography{mybib}{}

\end{document}